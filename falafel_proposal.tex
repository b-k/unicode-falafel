\documentclass[a4paper,10pt]{article}
%In the preamble section include the arabtex and utf8 packages
\usepackage{arabtex}
\usepackage{utf8}
\usepackage{url,graphicx}

%remove ugly boxes from pdf-embedded links.
\usepackage{hyperref,xcolor}
\hypersetup{%
      colorlinks=false,% hyperlinks will be black
            linkbordercolor=black,% hyperlink borders will be red
              pdfborderstyle={/S/U/W 0}% border style will be underline of width 1pt
}

\begin{document}
%start encoding to unicode
%Note that your layout must support arabic text when compiling
\setcode{utf8}




\title{Proposal to Add Emoji Symbol for {\sc Falafel} to Unicode}
\author{Ben Klemens\\ {\tt ben@klemens.org}}
\date{\today}
\maketitle

\begin{abstract}
This proposal requests the addition of a {\sc falafel} emoji to a future version of the
Unicode Standard.  The emoji food set is lacking in Middle Eastern and North African
foods, and in unambiguously vegetarian, kosher, or halal foods outside of fruits and
vegetables. A falafel emoji addresses these gaps.
\end{abstract}


\section{Introduction}

\begin{figure}
\begin{center}
\includegraphics[width=1.5in]{falafel.png}
\end{center}
\caption{Sample emoji rendering by Aphee Messer}
\label{apheefig}
\end{figure}

{\sc Falafel} is a ball of fried chickpeas or fava beans, and is popular in the Middle East
and North Africa, and increasingly throughout the world.  One of its key advantages is
that it accommodates a range of food preferences: given reasonable care in preparation,
it is vegan and vegetarian, halal, and kosher (for the remainder of this proposal, ``VHK'')

The proposed emoji would picture three falafel balls, with one in the center split open to
reveal the green interior of the ball. Figure \ref{apheefig} presents a sample rendering.

The food category in the emoji set includes a wide range of fruits and vegetables, but
very little else that is VHK. For example, cheeses may not be VHK due to rennet, and the
Japanese origins of emoji have led to a wide range of explicitly or likely
shellfish-based food emoji. The current emoji set has good representation of foods
from the American and East Asian regions, but very little from the Middle East and
North Africa.

\section{Expected Usage}

\subsection{Frequency}\label{freqsec}

\begin{figure}
\begin{center}
\includegraphics[width=3.5in]{trends.png}
\end{center}
\caption{In English, ``Falafel'' is as common as ``croissant'' in Google's database.}
\label{engplot}
\end{figure}


Figure \ref{engplot} shows frequencies of various food terms in Google's English-language
database of user searches.  In this data set, the usage of ``falafel'' is roughly as popular as
``croissant.'' 
The figure includes ``sweet potato,'' which is more popular
in Google's database but clearly seasonal, while Falafel shows consistent usage.
Some of the emoji related to Japanese food, {\sc rice ball} and {\sc bento box} are also
presented for comparison, though they may not be the ideal food for comparison to falafel.

To preserve the scale, ``rice'' was omitted from the plot of English search terms:
being a basic staple, it was searched on Google 17 times more often than ``falafel.''

\begin{figure}
\begin{center}
\includegraphics[width=3.5in]{atrends.png}
\end{center}
\caption{In Arabic, ``Falafel'' competes with basic staples in search popularity.}
\label{aplot}
\end{figure}

But among Arabic search terms, falafel is in the same range as basic staples. 
Figure \ref{aplot} shows search frequencies in Google's database in Arabic.
``Rice'' (\<رز>, in red) appears in the database of searches only three times as often as
falafel (\<فلافل>, in blue).  Another staple, ``bread'' (\<خبز>, in purple) is only 70\% more
popular than falafel at the end of this period. Falafel seems slightly more popular than ``kebab'' (\<كباب>,
in yellow), while ``croissant'' (\<كرواسون>, in green) seems largely unknown.

Trends in image searches are presented in the appendix in Figures \ref{engpicplot} and \ref{apicplot}.
The results are largely similar, but rice is somewhat more common in Arabic image searches
and English speakers are enamored of croissant photos.


\begin{figure}
\hspace{-2cm}
\begin{tabular}{rr}

\begin{tabular}{l|rrr}
{\bf English, count} & Bing & Google & YouTube \\\hline
Falafel  & 6,460 & 25,000 & 301\\
Rice    & 21,900 & 607,000 & 13,500\\
Sweet potato & 13,700 & 46,200 & 786\\
Croissant & 12,100 & 91,800 & 537 \\
Bento box & 2,160 & 8,760 & 229 \\
Rice ball & 926 & 1,010 & 154\\
\end{tabular}

&

\begin{tabular}{l|rrr}
{\bf English, scaled} & Bing & Google & YouTube \\\hline

Falafel  &1.00& 1.00 & 1.00 \\
Rice    &3.39&24.28 & 44.85 \\
Sweet potato &2.12& 1.85 & 2.61 \\
Croissant &1.87& 3.67 & 1.78 \\
Bento box &0.33& 0.35 & 0.76 \\
Rice ball &0.14& 0.04 & 0.51
\end{tabular}
\\
\phantom{Filler}\\

%Couldn't get Arabic working in tables...
\begin{tabular}{l|rrr}
{\bf Arabic} & Bing & Google & YouTube \\\hline
%\<ﻑﻼﻔﻟ>
Falafel & 301  & 5,670 & 460\\
%\<ﺃﺭﺯ>
Rice & 1,690& 10,400 & 595\\
%\<ﺦﺑﺯ>
Bread  & 601& 72,200   & 703\\
%\<ﻚﺑﺎﺑ>
Kebab & 303 & 2,650   & 625\\
%\<ﻙﺭﻭﺎﺳﻮﻧ>
Croissant & 62.2 & 405 & 63.7
\end{tabular}

&
\begin{tabular}{l|rrr}
{\bf Arabic} & Bing & Google & YouTube \\\hline
Falafel & 1.00 & 1.00  & 	1.00\\
Rice & 5.61 & 1.83	 &  1.29\\
Bread & 2.00 & 12.73 &    1.53\\
Kebab & 1.01 & 0.47	 &  1.36\\
Croissant & 0.21 & 0.07	 &  0.14
\end{tabular}

\end{tabular}
\caption{Search engine hits in thousands, from three search engines. Counts are given on
the left, and as a percent of the falafel count to the right.}
\label{counttab}

\end{figure}

Figure \ref{counttab} shows the search result counts in three search engines, in English
and Arabic. Apart from Arabic Google's anomalous count for bread, the results largely
follow those from the trend lines: in English, falafel result counts are behind but
on the same order of magnitude as sweet potatoes and croissants, while in Arabic, the
falafel result counts are behind but on the same order of magnitude as rice and bread.


\subsection{Use in sequences}\label{seqsec}
{\sc falafel} as street food is served as a wrap or in a flatbread, so sequencing it before {\sc stuffed flatbread}
facilitates the {\sc falafel}
+ {\sc stuffed flatbread} pairs, which could transform the wrap and flatbread emoji
into unambiguous representations of a falafel wrap or pita. {\sc falafel} + {\sc salad} is
another common menu item. Some restauarants serve a falafel burger as a vegetarian option,
which users might represent via {\sc falafel} + {\sc hamburger}.


\subsection{Image distinctiveness}
The ideal falafel is briefly fried so that the exterior is brown, but the interior remains green. We believe, from
our conversations, that we should show three falafel balls, and one of the of the balls should be cut open to show
the distinctive green color. The brown-to-green pattern is distinctive and easily recognized even in small fonts.
However, we leave it to the vendors to choose the most effective designs.

\subsection{Completeness}
{\sc Falafel} would be the first VHK Middle Eastern food represented in emoji.

Middle Eastern food has almost no representation in emoji. Distinctive foods
commonly found in the melting pot of Middle Eastern cuisine such as hummus, tahini, shakshuka or
baklava are missing, for various reasons.  Döner kebab has an emoji in the form
of {\sc stuffed flatbread}. The proposal itself, ``UTC document
L2/15-084,''\footnote{\url{http://www.unicode.org/L2/L2015/15084-kebab.pdf}} makes
only brief reference to its Turkish origins and instead bills it as ``Germany's most
favorite fast food snack;'' the final emoji is deliberately designed to generalize the
image to cover cuisine around the world.

\section{Selection Factors for Exclusion}

\subsection{Overly specific}

{\sc Falafel} has a level of specificity comparable to many other emoji, such as
{\sc croissant}, {\sc broccoli}, or {\sc sweet potato}.
%And it is as prominent for a particular
%region of the world which is currently underrepresented.

\subsection{Open ended}

Falafel recipes are largely uniform, so there is no need for additional emoji for
different falafel subtypes. One can find recipes for flatter discs or donut-shaped
falafel, but we feel that these are relatively uncommon and the traditional ball is
sufficient to express the idea of falafel.
% In addition, in our quick unscientific poll of Middle
%Easterners, falafel ranks among the highest for both being culturally relevant, and
%visually distinctive. In contrast, 
{\sc Hummus} has been previously proposed and rejected
by the Unicode Technical Committee. The universe of iconic and visually distinctive
foods in the Middle East is fairly scoped.

\subsection{Already Representable}

Falafel is often served in the form of a stuffed pita, but the {\sc stuffed flatbread}
emoji is designed to be ambiguous about its contents---and even whether the flatbread is a
pita at all, or a frybread or focaccia. ``Emoji Additions Tranche 6: More
Popular Requests and Gap Filling''\footnote{UTC document L2/15 195R2, \url{https://www.unicode.org/L2/L2015/15195r2-emoji-add-tranche6.pdf}} does propose ``falafel'' as an alias,
but the proposal describes ``ingredients, such as meat, vegetables, and condiments,''
and the proposed character in that document shows brown strips that can not be falafel.
As above, a key feature of falafel is that it is unambiguously VHK.  A VHK eater who
answers the question {\em what would you like for dinner?} with {\sc stuffed flatbread}
has no idea what he or she will get.

Falafel is not married to pita. As street food, it is often served in lafah,
a thin wrap producing a dish with closer resemblance to a burrito than a
stuffed pita. One may also find falafel on a mezze plate along with other small
dishes. Expressing the latter use with existing emoji, for example via {\sc stuffed
flatbread + fork and knife with plate}, may leave significant ambiguity that the author
is writing about flatbread containing primarily falafel, minus the flatbread.

\subsection{Transient}

The plot above from Google's database shows consistent usage of ``falafel'' in the
English-speaking world since 2004. Pre-Internet, falafel is old enough that its origins
have been lost. Claims include that its origin traces back to {\em ta'amia}, a fava-bean
based fritter perhaps originating among Christian Copts of Alexandria in centuries past,
or even the times of the pharaohs.

\subsection{Other exclusion factors}
The {\sc falafel} emoji is not a logo, brand, or other excluded category. 

Section \ref{freqsec} compared falafel's popularity to that of foods represented by early
emoji such as bento boxes and rice balls, but these comparisons are only one part of the
overall range, and the argument does not hinge on those comparisons.

No exact image is required.


\section{Location on the emoji keyboard}

As per Section \ref{seqsec}, falafel is well-paired with wraps and flatbreads, so 
sequencing it before {\sc stuffed flatbread}
facilitates the {\sc falafel} + {\sc stuffed flatbread} pairs.

\vfill
\paragraph{Thanks} This document owes a debt to Jennifer 8. Lee for editing and extensive
advice, Ronit Klemens for her expertise in Middle Eastern cuisine, and Aphee Messer for her visualizations.

\eject
\section*{Appendix}
The specifications for emoji proposals require screen shots of searches in several
search engines, and plots Google Trends counts for images. These are provided here
for reference. The counts are transcribed from here to Table \ref{counttab}. As per
the specification, these searches were done in a browser's privacy mode


\begin{figure}[!hb]
\begin{center}
\includegraphics[width=3.5in]{etrends-pics.png}
\end{center}
\caption{Counts of pictures in Google's English database, using the same color key as
Figure \ref{engplot}. Notably, falafel is in blue. Croissants are very photogenic.}
\label{engpicplot}
\end{figure}


\begin{figure}[!hb]
\begin{center}
\includegraphics[width=3.5in]{atrends-pics.png}
\end{center}
\caption{Counts of pictures in Google's Arabic database, using the same color key as Figure \ref{aplot}. Rice (red) appears more often than in
the search result count; falafel (blue) less often, but still commensurate with kebabs (yellow).}
\label{apicplot}
\end{figure}

\def\ss#1#2#3{\includegraphics[width=#3]{search_shots/#1-#2.png}}
\def\row#1#2{#1&\ss{b}{#2}{2.5cm}& \ss{g}{#2}{3.4cm}& \ss{y}{#2}{4.5cm}\\\hline}

\begin{figure}
\begin{center}
\begin{tabular}{l|c|c|c|}
 & Bing & Google &YouTube\\
\hline
\row{Falafel}{falafel}
\row{Rice}{rice}
\row{Sweet potato}{sweet_potato}
\row{Croissant}{croissant}
\row{Bento box}{bento_box}
\row{Rice ball}{rice_ball}
\end{tabular}
\end{center}
\caption{Screen shots of English language searches}
\label{enproof}
\end{figure}

\def\ss#1#2#3{\includegraphics[width=#3]{search_shots/#1-#2.png}}
\def\arow#1{#1&\ss{b-a}{#1}{2.5cm}& \ss{g-a}{#1}{3.4cm}& \ss{y-a}{#1}{4.5cm}\\\hline}

\begin{figure}
\begin{center}
\begin{tabular}{l|c|c|c|}
 & Bing & Google &YouTube\\
\hline
%\<ﻑﻼﻔﻟ> 
\arow{falafel}
%\<ﺃﺭﺯ>
\arow{rice}
%\<ﺦﺑﺯ> 
\arow{bread}
%\<ﻚﺑﺎﺑ> 
\arow{kebab}
%\<ﻙﺭﻭﺎﺳﻮﻧ> 
\arow{croissant}
\end{tabular}
\end{center}
\caption{Screen shots of Arabic language searches}
\label{arproof}
\end{figure}

\end{document}
